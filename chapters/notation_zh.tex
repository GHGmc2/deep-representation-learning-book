
\chapter*{符号表示}

%\DP{后续可以充实。如需添加任何符号,请在此处注明。}

\section*{线性代数符号}
\begin{itemize}
    \item 标量为(大写或小写)非粗体字母,例如:\(x\)、\(X\)。
    \item 向量为小写粗体字母,例如:\(\vx\)、\(\vpi\)。向量的分量记作 \(x_{i}\) 或 \(\pi_{i}\),也可记作 \((\vx)_{i}\)。
    \item 矩阵为大写粗体字母,例如:\(\vX, \vPi\)。矩阵的元素记作 \(X_{ij}\) 或 \(\Pi_{ij}\),也可记作 \((\vX)_{ij}\)。
    \item 随机变量与确定性变量使用相同的符号表示。
    \item 向量的 \(\ell_{p}\) 范数记作 \(\norm{\vx}_{p}\),其中 \(p \in \{0\} \cup [1, \infty]\)。
    \item 矩阵的欧几里得算子范数记作 \(\norm{\vA}\)。
    \item 矩阵的弗罗贝尼乌斯范数记作 \(\norm{\vA}_{F}\)。
    \item 向量的欧几里得内积记作 \(\ip{\vx}{\vy}\)。
    \item 矩阵的弗罗贝尼乌斯内积记作 \(\ip{\vX}{\vY}\)。
    \item 转置记作 \(\top\),例如:\(\vA^{\top}\)。 %
    伴随矩阵记作 $\adj$,例如:$\vA\adj$。 %
    伪逆记作 \(\dagger\),例如:\(\vA^{\dagger}\)。
    \item 到集合(例如 \(S\))上的投影记作 \(\proj_{S}\)。
    \item 正交矩阵集记作 \(\O(m, n)\) 或 \(\O(n) = \O(n, n)\)。
    \item 对称矩阵集记作 \(\Sym(n)\),对称半正定(PSD)矩阵集记作 \(\PSD(n)\),对称正定(PD)矩阵集记作 \(\PD(n)\)。
    \item 球面记作 \(\Sphere^{n - 1} \subseteq \R^{n}\),球体记作 \(\Ball^{n} \subseteq \R^{n}\)。
    \item 矩阵的秩记作 \(\rank(\vX)\),迹记作 \(\tr(\vX)\),行列式记作 \(\det(\vX)\),对数行列式记作 \(\logdet(\vX)\)。
    \item 对称矩阵 \(\vX \in \Sym(n)\) 的特征值记作 \(\lambda_{i}(\vX)\),满足 \(\lambda_{1}(\vX) \geq \cdots \geq \lambda_{n}(\vX)\)。
    \item 秩为 \(r\) 的矩阵 \(\vX \in \R^{n \times d}\) 的奇异值满足 \(\sigma_{1}(\vX) \geq \cdots \geq \sigma_{r}(\vX) > 0\)。按照惯例,我们设定 \(\sigma_{r + 1}(\vX) = \sigma_{r + 2}(\vX) = \cdots = 0\)。
    \item 逐元素乘法(即哈达玛积)记作 \(\vX \hada \vY\)。逐元素平方记作 
    $\vX^{\hada 2}$。
    \item 克罗内克积记作 $\vX \kron \vY$。迭代克罗内克积记作 $\vX^{\kron 2}$。
    \item 逐元素函数应用记作 \(f[\vX]\),即使用方括号。
    \item 全一向量/矩阵记作 \(\vone\)。全零向量/矩阵记作 \(\vzero\)。单位矩阵记作 \(\vI\)。
\end{itemize}

\section*{概率论符号}

\begin{itemize}
    \item 概率记作 \(\Pr\)。
    \item 期望值记作 \(\Ex\)。 
    \item 协方差记作 \(\Cov\)。 
    \item 相关系数记作 \(\Corr\)。
    \item 可测空间 \(\vX\) 上的概率分布集记作 \(\Delta(\vX)\)。(当 \(\cX = [n]\) 时,此为单纯形)。
\end{itemize}

\section*{机器学习符号}

\begin{itemize}
    \item \(\vx \in \R^{D} = \cX\) 是向量值数据随机变量(当每个样本为独立个体时)。\(\vX \in \R^{N \times D} = \cX\) 是矩阵值数据随机变量(当每个样本为一个词元集时)。
    \item \(\vz \in \R^{d} = \cZ\) 是向量值特征随机变量。\(\vZ \in \R^{n \times d} = \cZ\) 是矩阵值特征随机变量。
    \item 映射 \(f \colon \cX \mapsto \cZ\) 为表示映射。\(\vZ = f(\vX)\) 是 \(\vX\) 的特征。
    \item 映射 \(g \colon \cZ \to \cX\) 为解码映射。\(\hat{\vX} = g(\vZ)\) 是 \(\vX\) 的自编码重构。 % \DP{Yi 喜欢用 hat 而不是 widehat。}
    \item 当 \(f\) 和 \(g\) 实现为神经网络时,不失一般性,它们具有相同的层数;我们将其写作 \(f = f^{L} \circ f^{L - 1} \circ \cdots \circ f^{2} \circ f^{1} \circ f^{\pre}\) 和 \(g = g^{\post} \circ g^{1} \circ g^{2} \circ \cdots \circ g^{L - 1} \circ g^{L}\)。通常,{\(g^{\ell} \circ f^{\ell} \approx \id\)}。
\end{itemize}
