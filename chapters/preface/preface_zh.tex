\providecommand{\toplevelprefix}{../..}  % necessary for subfile bibliography + figures compilation to work, do not move this after documentclass
\documentclass[../../book-main.tex]{subfiles}
\usepackage[UTF8]{ctex}
\begin{document}

\chapter*{序言}

\begin{center}
%\hfill    
``{\em 条条大路通罗马}。''

%$~$ \hfill --- 
\end{center}
\vspace{5mm}

本书揭示并研究了几乎所有现代(人工智能)实践背后一个共同而根本的问题。即,{\em 如何在高维空间中有效且高效地学习数据的低维分布,并将其转换为一个紧凑且结构化的表示?}对于一个智能系统,无论是自然的还是人工的,这种表示通常可以被看作是从外部世界感知的数据中学习到的{\em 记忆}或{\em 知识}。

本教科书旨在为{\em 高年级本科生和研究生新生}系统地介绍学习此类数据分布的(深度)表示的数学和计算原理。本书的主要先修知识是本科水平的线性代数、概率/统计和优化。若能熟悉信号处理、信息论和反馈控制的基本概念,将有助于您更好地理解本书内容。

撰写本书的主要动机在于,过去数年间,本书作者及诸多同仁在建立一个原则性、严谨的方法以理解深度神经网络乃至智能本身这一领域取得了巨大进展。这种新方法所倡导的演绎法,与当前人工智能实践背后主流的、主要基于归纳和试错的方法,形成了鲜明的对比,同时又具有高度的互补性。对于如此发展起来的强大人工智能模型和系统,我们缺乏深刻的理解,这在社会上引发了越来越多的炒作和恐慌。我们相信,现在比以往任何时候都更需要认真地尝试建立一种有原则的方法来理解智能。本书的一个首要目标,是提供坚实的理论和实验证据,以表明将智能作为一个科学和数学课题来研究,如今已成为可能。

在更技术的层面上,本书提出的框架有助于弥合两种方法之间的长期鸿沟:一种是主要基于解析几何、代数和概率模型(例如,子空间和高斯分布)来为数据结构建模的经典方法;另一种是基于经验设计的非参数模型(例如,深度网络)的“现代”方法。事实证明,如果我们认识到这两种看似分离的方法都在试图建模和学习目标数据分布中的{\em 低维}结构,那么它们的统一就变得可能,甚至是自然的。它们只是追求、表示和利用低维结构的不同方式而已。从这个角度来看,许多在不同时期、不同领域独立发展起来的、看似无关的计算技术,现在都可以在一个共同的计算框架下得到更好的理解,并且或许从今以后可以被放在一起研究。正如我们将在本书中看到的,这些技术包括但不限于信息论和编码理论中发展的有损压缩编解码、信号处理和机器学习中的扩散与去噪,以及优化中的增广拉格朗日方法等延拓技术。

我们相信,对于那些真正希望澄清关于深度神经网络和(人工智能)的谜团与误解的读者来说,本书所呈现的统一概念和计算框架将具有巨大的价值。此外,该框架旨在为读者提供指导原则,以在未来开发出更卓越、更真正智能的系统。更具体地说,本书的主要技术内容组织如下:
\begin{itemize}
\item 我们将从经典且最基础的主成分分析(PCA)、独立成分分析(ICA)和字典学习(DL)模型讲起,这些模型假设目标低维分布具有线性和独立的结构。通过这些简单而理想化的模型,我们将介绍学习低维分布最基本、最重要的思想。

\item 为了将这些经典模型及其解法推广到一般的低维分布,我们引入一个学习此类分布的普适计算原理:{\em 压缩}。正如我们将看到的,数据压缩为所有看似不同的经典与现代的分布或表示学习方法提供了一个统一的视角,包括降维、用于去噪的分数匹配,以及用于有损压缩的熵或编码率降低。

\item 在这个统一的框架内,现代的深度神经网络(DNN),如ResNet、CNN和Transformer,都可以被数学地解释为(展开的)优化算法,通过降低编码长度/码率或增加信息量来迭代地实现更好的压缩和更优的表示。这个框架不仅有助于解释迄今为止经验性设计的深度网络架构,还能启发设计出可能更简洁、更高效的新架构。

\item 此外,为确保学习到的数据分布表示是正确且一致的,包含编码和解码(例如,通过去噪和扩散)的{\em 自动编码}架构变得必不可少。为了使学习系统能够全自动且持续地运行,我们将引入一个强大的{\em 闭环转录框架},该框架使系统能通过编码器和解码器之间的极小极大博弈进行自我纠正,从而实现自我提升。

\item 我们还将研究如何利用这样学习到的数据分布和表示作为一个强大的先验,以进行贝叶斯推断,并促进许多依赖于条件估计、补全和生成的实际任务,例如处理图像等现实世界中的高维数据。

\item 最后但同样重要的是,为了连接理论与实践,我们将逐步演示如何利用大规模数据集(包括图像和文本)有效且高效地学习数据分布的(深度)表示,并将其应用于许多实际应用中,例如图像分类、图像补全、图像分割、图像生成,以及针对文本数据的类似任务。
\end{itemize}

总而言之,本书呈现的技术内容旨在建立起经典解析方法与现代计算方法之间、简单参数模型与深度非参数模型之间、多样的归纳实践与统一的第一性原理演绎框架之间的紧密概念和技术联系。尽管这些方法在不同时期、不同领域各自发展,但它们都致力于实现一个共同的目标:{\em 探寻并利用高维数据分布内在的低维结构。}为此,本书将带领我们走过一段完整的旅程,从理论构建,到数学验证,再到计算实现,最终到实际应用。
\end{document}
