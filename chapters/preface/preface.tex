\providecommand{\toplevelprefix}{../..}  %
\documentclass[../../book-main.tex]{subfiles}

\begin{document}

\chapter*{Preface}

\begin{center}
``{\em All roads lead to Rome}.''

\end{center}
\vspace{5mm}

This book reveals and studies a common and fundamental problem behind almost all modern practices of (artificial) intelligence. That is, {\em how to effectively and efficiently learn a low-dimensional distribution of data in a high-dimensional space and then transform the distribution to a compact and structured representation?} For any intelligent system, natural or man-made, such a representation can be generally regarded as a {\em memory} or {\em knowledge} learned from data sensed from the external world.

This textbook aims to provide a systematic introduction to the mathematical and computational principles for learning (deep) representations of such data distributions for {\em senior undergraduate students and beginning graduate students}. The main prerequisites for this book are undergraduate linear algebra, probability/statistics, and optimization. Some familiarity with basic concepts from signal processing (sparse representation and compressed sensing in particular), information theory, and feedback control would enhance your appreciation. 

The main motivation for writing this book is that there have been tremendous developments in the past several years, by the authors and many colleagues, that aim to establish a principled and rigorous approach to understand deep neural networks and, more generally, intelligence itself. The deductive methodology advocated by this new approach is in direct contrast, and highly complementary, to the dominant methodology behind current practices of artificial intelligence, which is largely inductive and trial-and-error. The lack of understanding about such powerful AI models and systems has led to increasing hype and fears in society. We believe that a serious attempt to establish a principled approach to understand intelligence is more needed than ever. An overarching goal of this book is to provide solid theoretical and experimental evidence showing that it is now possible to study intelligence as a scientific and mathematical subject. Hence, one may view this book as a first attempt to develop {\em a Mathematical Theory of Intelligence}.   

At the technical level, the theoretical framework presented in this book helps reconcile a long-standing gap between the classical approach to modeling data structures that are mainly based on analytical geometric, algebraic, and probabilistic models (e.g., subspaces, Gaussians, and equations) and the ``modern'' approach that is based on empirically designed non-parametric models (e.g., deep networks). As it turns out, a unification of the two seemingly separate methodologies becomes possible and even natural if one realizes that they all try to model and learn {\em low-dimensional} structures in the data distribution of interest. They are merely different ways to pursue, represent, and exploit the low-dimensional structures. From this perspective, even many seemingly unrelated computational techniques, developed independently in separate fields at different times, can now be better understood under a common computational framework and probably can be studied together from now on. As we will see in this book, these techniques include but are not limited to lossy compressive encoding-decoding developed in information theory and coding theory, diffusion and denoising in signal processing and machine learning, and continuation techniques such as augmented Lagrangian methods for constrained optimization. 

We believe that the unified conceptual and computational framework presented in this book will be of great value to readers who truly want to clarify mysteries and misunderstandings about deep neural networks and (artificial) intelligence. Furthermore, the framework is meant to provide readers with guiding principles for developing significantly better and truly intelligent systems in the future. More specifically, besides a general introduction (chapter), the main technical content of the book will be organized as six closely related topics (chapters):
\begin{enumerate}
\item We will start with the classical and most basic models of Principal Component Analysis (PCA), Independent Component Analysis (ICA), and Dictionary Learning (DL), which assume that the low-dimensional distributions of interest have linear and independent structures. From these simple idealistic models that are well studied and understood in signal processing and compressed sensing, we will introduce the most basic and important ideas for how to learn low-dimensional distributions.

\item To generalize these classical models and their solutions to general low-dimensional distributions, we introduce a universal computational principle for learning such distributions: {\em compression}. As we will see, data compression provides a unifying view of all seemingly different classic and modern approaches to distribution or representation learning, including dimensionality reduction, entropy minimization, score matching for denoising, and lossy compression with rate distortion. 

\item Within this unifying framework, modern Deep Neural Networks (DNNs), such as ResNet, CNN, and Transformer, can all be mathematically interpreted as (unrolled) optimization algorithms that iteratively achieve better compression and better representations by reducing coding length/rate or gaining information. Not only does this framework help explain empirically designed deep network architectures thus far, it also leads to new architecture designs that can be significantly simpler and more efficient.

\item Furthermore, to ensure that the learned representation for a data distribution is correct and consistent, the {\em auto-encoding} architectures that consist of both encoding and decoding become necessary. In order for a learning system to be fully automatic and continuous, we will introduce a powerful {\em closed-loop transcription framework} that enables an auto-encoding system to self-correct and thus self-improve via a minimax game between the encoder and decoder.  

\item We will also study how the learned data distribution and representation can be utilized as a powerful prior or constraint to conduct Bayesian inference that facilitates almost all types of tasks and settings that are popular in the practice of modern artificial intelligence, including conditional estimation, completion, and generation of real-world high-dimensional data such as images and texts. 

\item Last but not least, to connect theory to practice, we will demonstrate step-by-step how to effectively and efficiently learn deep representations of low-dimensional data distributions with large-scale datasets, including both images and texts, and use them in many practical applications such as image classification, image completion, image segmentation, image generation, and similar tasks for text data. 
\end{enumerate}

To summarize, the technical content presented in this book establishes strong conceptual and technical connections between the classical analytical approach and the modern computational approach, between simple parametric models and deep non-parametric models, between diverse inductive practices and a unified deductive framework from first principles. We will reveal that many seemingly unrelated or even competing approaches, though developed in separate fields at different times, all strive to achieve a common objective: 
\begin{quote}
\centering{\em pursuing and exploiting intrinsic low-dimensional distributions of high-dimensional data.}    
\end{quote}
To this end, the book will take us through a complete journey from theoretical formulation to mathematical verification to computational realization to practical applications.




\end{document}
