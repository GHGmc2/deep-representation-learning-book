\providecommand{\toplevelprefix}{../..}  % necessary for subfile bibliography + figures compilation to work, do not move this after documentclass
\documentclass[../../book-main.tex]{subfiles}

\begin{document}

\chapter*{致谢}
% This book is primarily based on research results that have been developed within the past eight years. Thanks to the generous startup funds from UC Berkeley (2018) and the University of Hong Kong (2023), Yi Ma was able to embark and focus on this new exciting research direction in the past eight years. Through these years, related to this research direction, Yi Ma and his research team at Berkeley have been supported by the following research grants:
本书主要基于过去八年中所得的研究成果。感谢加州大学伯克利分校(2018)和香港大学(2023)慷慨地提供启动经费,使马毅在过去八年能够投身并专注于这一令人振奋的新研究方向。在这些年里,围绕该研究方向,马毅及其在伯克利的团队获得了以下研究项目的资助:

\begin{itemize}
    \item 由 Simons Foundation 和 National Science Foundation (DMS grant \#2031899)共同资助的多个大学联合开展的THEORINET project for the Foundations of Deep Learning项目;%The multi-university {\em THEORINET} project for the Foundations of Deep Learning, jointly funded by the Simons Foundation and the National Science Foundation (DMS grant \#2031899);
    \item 由 Office of Naval Research (grant N00014-22-1-2102) 资助的Closed-Loop Data Transcription via Minimaxing Rate Reduction项目;% The {\em Closed-Loop Data Transcription via Minimaxing Rate Reduction} project funded by the Office of Naval Research (grant N00014-22-1-2102); 
    \item 由 National Science Foundation (CISE grant \#2402951) 资助的Principled Approaches to Deep Learning for Low-dimensional Structures项目。% The {\em Principled Approaches to Deep Learning for Low-dimensional Structures} project funded by the National Science Foundation (CISE grant \#2402951).
\end{itemize} 

% This book would have not been possible without the financial support for these research projects. The authors have drawn tremendous inspiration from research results by colleagues and students who have been involved in these projects.
如果没有这些研究项目的资金支持,本书将无法完成。作者从参与这些项目的同事和学生的研究成果中获得了巨大的启发。


本书原版以英文撰写。在翻译为中文的过程中,我们充分利用了多个人工智能模型的辅助:秘塔科技公司的 LaTeX 翻译模型生成了中文初稿,然后我们用Gemini 2.5-Pro 对部分章节进行了编辑和优化。在确保中英文版本公式、LaTeX 语法及专有名词一致性的前提下,人工智能模型完成了对全文文字内容的初步翻译。随后,我们对中文手稿进行了人工微调,对语法、排版及行文风格进行了系统优化,以呈现更加自然、流畅的阅读体验,同时兼顾学术严谨性和可读性。当然,我们今后还会对中英文版本进行不断修改。

另外,在本书的网页版:
\begin{center}
    \url{https://ma-lab-berkeley.github.io/deep-representation-learning-book/}
\end{center}
我们为方便读者学习本书开发并训练了基于大语言模型的在线聊天机器人。读者可以随时对书籍中的内容以及相关领域的知识进行查询交互。该模型由香港大学博士生褚天哲在忆生科技公司实习时完成开发,通过使用英文原版书本内容以及相关参考文献对Qwen2.5-7B-Instruct进行微调而成。其主要功能包括:
\begin{itemize}
    \item 内容问答:针对书中章节,提供精准回答,辅助读者理解内容。

    \item 概念解读:对书中专业术语和关键概念进行详细解释。

    \item 示例与应用:提供相关实例和应用场景,帮助读者将理论与实践结合。
\end{itemize}

最后,我们衷心感谢忆生科技与香港大学上海智能计算研究院为本书的开源提供的技术与算力支持。这些资源不仅保障了书中部分科研工作的顺利开展和实现,也使得本书的中文翻译以及人工智能辅助工具的开发得以顺利完成。% 同时,我们也感谢以下成员在本书撰写过程中提出的宝贵意见和反馈:曲庆(密西根大学)。
\end{document}